\documentclass{article}

\begin{document}
\noindent {\bf {\large Group Members: Stephen Kuo, Derek Mui}}

\vspace{.2in}

\noindent {\bf {Homework 1}}

\vspace{.2in}
\noindent 1) In each of the following situations, indicate whether $f  = O(g)$, or $f = \Omega(g)$, 
\indent or both (in which case $f = \Theta(g)$).

\vspace{.1in}
{\bf Answer:} \\*
%1
\indent (a) $f(n) = n - 100$ and $g(n) = n - 200$ \\*
\indent \indent Both are $O(n)$, so $f = \Theta(g)$ \\*
%2
\indent (b) $f(n) = n^{1/2}$ and $g(n) = n^{2/3}$\\*
\indent \indent Since they're both power s of $n$, compare the powers. $\frac{1}{2} < \frac{2}{3}$ so $f = O(g)$ \\*
%3
\indent (c) $f(n) = 100n + $log $n$ and $g(n) = n + ($log $n)^2$\\*
\indent \indent They are both $O(n)$ so $f = \Theta(g)$ \\*
%4
\indent (d) $f(n)  = n$ log $n$ and $g(n) = 10n$ log $10n$\\*
\indent \indent They are both $O($log $n)$ so $f = \Theta(g)$ \\*
%5
\indent (e) $f(n) = $log $2n$ and $g(n) = $log $3n$\\*
\indent \indent They are both $O($log $n)$ so $f = \Theta(g)$ \\*
%6
\indent (f) $f(n) = 10$ log $n$ and $g(n) = $log $(n^2)$ \\*
\indent \indent Both are $O($log $n)$ so $f = \Theta(g)$ \\*
%7
\indent (g) $f(n) = n^{1.01}$ and $g(n) = n log^2n$\\*
\indent \indent If both sides are divided by $n$ we then need to compare $n^{0.01}$ and $log^2n$. \\*
\indent \indent Ultimately, the power function wins, so $f = \Omega(g)$\\*
%8
\indent (h) $f(n) = \frac{n^2}{logn}$and $g(n) = n($log $n)^2$\\*
\indent \indent Divide both sides by $\frac{n}{logn}$ and we will only need to compare $n$ and $(logn)^3$. \\*
\indent \indent The result is $f = \Omega(g)$\\*
%9
\indent (i) $f(n) = n^0.1$ and $g(n) = ($log $n)^{10}$\\*
\indent \indent Similar to problems g and h, $f = \Omega(g)$\\*
%10
\indent (j) \\*
%11
\indent (k) \\*
%12
\indent (l) \\*
%13
\indent (m) \\*
%14
\indent (n) \\*
%15
\indent (o) \\*
%16
\indent (p) \\*
%17
\indent (q) \\*


\vspace{.5in}
\noindent 2) Show that, if \textit{c} is a positive real number, then \textit{g(n)} = 1 + \textit{c} + $c^2$ + . . . + 
\indent$c^n$ is:

\vspace{.1in}
\indent (a) $\Theta$(1) if $c < 1$

\indent (b) $\Theta$(1) if $c = 1$

\indent (c) $\Theta$(1) if $c > 1$

\vspace{.1in}
{\bf Answer:}
\\* \indent If c = 1, $g(n) = 1 + 1 + . . .  + 1 = n + 1 = \Theta(n)$. Otherwise:
\\* \\* \indent \indent \indent $g(n) = \frac{c^{n + 1} - 1}{c - 1} = \frac{1 - c^{n + 1}}{1 - c}$ \\* 

\indent If $c < 1$, then $1 - c < 1 - c^{n + 1} < 1$. So, $1 < g(n) < \frac{1}{1 - c}$. Thus, $g(n) = \Theta(1)$ \\*

\indent If $c > 1$, then $c^{n + 1} > c^{n + 1} - 1 > c^n$. So, $\frac{c^n}{1 - c} < g(n) < \frac{c}{1 - c} * c^n$. \\*
\indent Thus, $g(n) = \Theta(c^n)$

\vspace{.5in}
\noindent 3) Determine the number of paths of length 2 in a complete graph of \textit{n} nodes.
\indent Give your answer in Big-\textit{O} notation as a function of \textit{n}.

\vspace{.1in}
{\bf Answer:}



\end{document}