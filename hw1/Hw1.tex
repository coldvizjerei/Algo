\documentclass{article}

\begin{document}
\begin{center}
\bf{\LARGE CS344: Design and Analysis of Computer Algorithms} \\*

\vspace{0.2in}
{\bf {\Large Homework 1}}
\end{center}

\vspace{.2in}

\noindent {\bf {\large Group Members: Stephen Kuo, Derek Mui}}

\vspace{.2in}
\noindent 1) In each of the following situations, indicate whether $f  = O(g)$, or $f = \Omega(g)$, 
\indent or both (in which case $f = \Theta(g)$).

\vspace{.1in}
{\bf Answer:} \\*
%1
\indent (a) $f(n) = n - 100$ and $g(n) = n - 200$ \\
\indent \indent Both are $O(n)$, so $f = \Theta(g)$ \\

%2
\indent (b) $f(n) = n^{1/2}$ and $g(n) = n^{2/3}$\\
\indent \indent Since they're both power s of $n$, compare the powers. $\frac{1}{2} < \frac{2}{3}$ so $f = O(g)$ \\

%3
\indent (c) $f(n) = 100n + $log $n$ and $g(n) = n + ($log $n)^2$\\
\indent \indent They are both $O(n)$ so $f = \Theta(g)$ \\

%4
\indent (d) $f(n)  = n$ log $n$ and $g(n) = 10n$ log $10n$\\
\indent \indent They are both $O($log $n)$ so $f = \Theta(g)$ \\

%5
\indent (e) $f(n) = $log $2n$ and $g(n) = $log $3n$\\
\indent \indent They are both $O($log $n)$ so $f = \Theta(g)$ \\

%6
\indent (f) $f(n) = 10$ log $n$ and $g(n) = $log $(n^2)$ \\
\indent \indent Both are $O($log $n)$ so $f = \Theta(g)$ \\

%7
\indent (g) $f(n) = n^{1.01}$ and $g(n) = n log^2n$\\
\indent \indent If both sides are divided by $n$ we then need to compare $n^{0.01}$ and $log^2n$. \\
\indent \indent Ultimately, the power function wins, so $f = \Omega(g)$\\

%8
\indent (h) $f(n) = \frac{n^2}{logn}$and $g(n) = n($log $n)^2$\\
\indent \indent Divide both sides by $\frac{n}{logn}$ and we will only need to compare $n$ and $(logn)^3$. \\
\indent \indent The result is $f = \Omega(g)$\\

%9
\indent (i) $f(n) = n^0.1$ and $g(n) = ($log $n)^{10}$\\
\indent \indent Similar to problems g and h, $f = \Omega(g)$\\

%10
\indent (j) $f(n) = (logn)^{logn}$ and $g(n) = \frac{n}{logn}$\\
\indent \indent The function $f(n) = n^{loglogn}$, thus $f = \Omega(g)$\\

%11
\indent (k) $f(n) = \sqrt{n}$ and $g(n) = (logn)^3$\\
\indent \indent Again, $f = \Omega(g)$\\

%12
\indent (l) $f(n) = n^{1/2}$ and $g(n) = 5^{\log_2 n}$\\
\indent \indent $g(n) = n^{\log_2 5} \approx  n^{2.32}$ thus, $f = O(g)$\\

%13
\indent (m) $f(n) = n2^2$ and $g(n) = 3^n$\\
\indent \indent $\lim_{n \to +\infty}$ $\frac{n * 2^n}{3^n} = \frac{n}{1.5^n} \Rightarrow f = O(g)$\\

%14
\indent (n) $f(n) = 2^n$ and $g(n) = 2^{n + 1}$\\
\indent \indent Here, $f = \Theta(g)$\\

%15
\indent (o) $f(n) = n!$ and $g(n) = 2^n$\\
\indent \indent It seems that $n! > \sqrt{2\pi n(\frac{n}{e})^n}$. Thus, $f = O(g)$\\

%16
\indent (p) $f(n) = (\log n)^{\log n}$ and $g(n) = 2^{({\log_2 n})^2}$\\
\indent \indent The function $f(n) = n^{\log \log n}$ and the function \\
\indent \indent $g(n) = (2^{\log_2 n})^{\log_2 n} = n^{\log_2 n}$. Thus, $f = O(g)$\\

%17
\indent (q) $f(n) = \sum\limits_{i=1}^n i^k$ and $g(n) = n^{k + 1}$\\
$f(n) = 1^k + 2^k + . . . + n^k \le n^k + n^k + . . . n^k = n*n^k = n^{k + 1} = g(n) \Rightarrow f(n) = O(g(n))$\\

\vspace{0.1in}
Also: \\
$f(n) = 1^k + 2^k + . . . + (\frac{n}{2})^k + (\frac{n}{2} + 1)^ k + . . . + n^k \ge \frac{n^k}{2^k} + \frac{n^k}{2^k} + . . . + \frac{n^k}{2^k} = \frac{n}{2} * \frac{1}{2^k} * n^k = n^{k + 1} * \frac{1}{2^{k + 1}} \Rightarrow f = \Omega(g)$ \\

\vspace{0.1in}
Thus, $f = \Theta(g)$


\vspace{.5in}
\noindent 2) Show that, if \textit{c} is a positive real number, then \textit{g(n)} = 1 + \textit{c} + $c^2$ + . . . + 
\indent$c^n$ is:

\vspace{.1in}
\indent (a) $\Theta$(1) if $c < 1$

\indent (b) $\Theta$(1) if $c = 1$

\indent (c) $\Theta$(1) if $c > 1$

\vspace{.1in}
{\bf Answer:}
\\*
\indent If c = 1, $g(n) = 1 + 1 + . . .  + 1 = n + 1 = \Theta(n)$. Otherwise: \\

\indent \indent \indent $g(n) = \frac{c^{n + 1} - 1}{c - 1} = \frac{1 - c^{n + 1}}{1 - c}$ \\

\indent If $c < 1$, then $1 - c < 1 - c^{n + 1} < 1$. So, $1 < g(n) < \frac{1}{1 - c}$. Thus, $g(n) = \Theta(1)$ \\

\indent If $c > 1$, then $c^{n + 1} > c^{n + 1} - 1 > c^n$. So, $\frac{c^n}{1 - c} < g(n) < \frac{c}{1 - c} * c^n$. \\
\indent Thus, $g(n) = \Theta(c^n)$

\vspace{.5in}
\noindent 3) Determine the number of paths of length 2 in a complete graph of \textit{n} nodes.
\indent Give your answer in Big-\textit{O} notation as a function of \textit{n}.

\vspace{.1in}
{\bf Answer:} If a graph has 3 vertices, then there are 3 paths of length 2 on that particular graph. Thus, that gives us 3$a \choose b$


\end{document}