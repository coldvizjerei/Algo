\documentclass{article}

\usepackage{amssymb}
\usepackage{amsmath}

\begin{document}
\begin{center}
\bf{\LARGE CS344: Design and Analysis of Computer Algorithms} \\*

\vspace{0.2in}
{\bf {\Large Homework 6}}
\end{center}

\vspace{.2in}

\noindent {\bf {\large Group Members: Stephen Kuo, Derek Mui}}

\vspace{.2in}
\noindent \textbf{\underline{4.1)}} Suppose Dijkstra's algorithm, is run on the graph, starting t node $A$.\\

\indent (a) Draw a table showing the intermediate distance values of all the nodes at each iteration of the algorithm \\
\indent (b) Show the final shortest path tree \\

\noindent \textbf{\underline{Solution:}}  \\
a) \\
\begin{tabular}{ | c c c c c c c c |}
  \hline
  A & B & C & D & E & F & G & H\\ \hline
  0 & $\infty$ & $\infty$ & $\infty$ & $\infty$ & $\infty$ & $\infty$ & $\infty$ \\
  0 & 1 & $\infty$ & $\infty$ & 4 & 8 & $\infty$ & $\infty$\\
  0 & 1 & 3 & $\infty$ & 4 & 7 & 7 & $\infty$\\
  0 & 1 & 3 & 4 & 4 & 7 & 5 & $\infty$\\
  0 & 1 & 3 & 4 & 4 & 7 & 5 & 8\\
  0 & 1 & 3 & 4 & 4 & 7 & 5 & 8\\
  0 & 1 & 3 & 4 & 4 & 6 & 5 & 6\\
  0 & 1 & 3 & 4 & 4 & 6 & 5 & 6\\
  0 & 1 & 3 & 4 & 4 & 6 & 5 & 6\\
  \hline
\end{tabular} \\

\vspace{.2in}
\noindent b) \\
\indent 1 \indent 2 \indent 1 \\
     A   $\rightarrow$    B         $\rightarrow$          C         $\rightarrow$        D\\
      $\downarrow$ 4 \indent \indent $\downarrow$ 2 \\
      E \indent F $\leftarrow$ G $\rightarrow$ H\\
\indent \indent 1 \indent \indent 1\\


\vspace{.2in}
\noindent \textbf{\underline{4.5)}} Often there are multiple shortest paths between two nodes of a graph. Give a linear-time algorithm for the following task. \\

\indent \textit{Input:} Undirected graph G = (V, E) with unit edge lengths nodes $u, v \in V$ \\
\indent \textit{Output:} The number of distinct shortest paths from $u$ to $v$ \\

\noindent \textbf{\underline{Solution:}}  \\
\indent Breadth-first search gives us one shortest path from u to v or to any other vertex reachable from u. To compute the number of shortest paths, we need to make an addition to the algorithm. \\
Assume that the distance from u to some vertex x is 10.  Then x has at least one neighbor $y_1$ whose distance from u is 9.  Let us say that there are two more neighbors $y_2$, $y_3$ at distance 9 from u. Any shortest path from u to x can be constructed by choosing $y \in \{y_1, y_2, y_3\}$ as the second last vertex, taking a shortest path from u to y, and adding the final edge (y, x). Each choice $y_1, y_2$, or $y_3$ gives a distinct set of paths. The total number of shortest paths from u to x is then the sum of the number of shortest paths from u to $y_1, y_2$, and $y_3$. Generally, once we know the number of shortest paths from u to all vertices at distance d, we can compute the number of shortest paths to the vertices at distance d+1 without actually enumerating the paths. \\

We use BFS to go through the vertices in the order of increasing distance from the start vertex. The algorithm  count-shortest-paths below performs a BFS starting from u. The number of shortest paths from u to any vertex x, denoted by paths [x], is initialized to 0, except that there is 1 shortest path of length 0 from u to u. On line 14, we accumulate paths [x] by summing paths [y] for all neighbors y of x such that y is on a shortest path between u and x. In the end, paths [v] contains the number of shortest paths from u to v, or zero if v is unreachable from u. \\

As a side effect, the algorithm finds the number of shortest paths from u to all vertices and not just to v , but this does not affect the worst-case run time. The run time of BFS is $O(|V| + |e|)$, and count-shortest-paths stays in the same bound, assuming that integer addition is constant-time. Note that we might need to break this assumption in practice because the number of shortest paths can grow exponentially in the number of vertices. \\

\noindent \textbf{Computing number shortest paths using BFS} \\
1 function count-shortest-paths(G, u, v); \\
2 for all x $\in$ V do \\
3 \indent dist[x] $\leftarrow \infty$; \\
4 \indent paths[x] $\leftarrow 0$ \\
5 end \\
6 ; \\
7 dist[u] $\leftarrow 0$; \\
8 paths[u] $\leftarrow 1$; \\
9 Q $\leftarrow$ [u]\\
10 while Q is not empty do \\
11 \indent x $\leftarrow$ EJECT(Q); \\
12 \indent for all edges (x, y) $\in$ E do \\
13 \indent \indent if dist[y] = dist[x] - 1 then \\
14\indent \indent \indent paths[x] $\leftarrow$ paths[x] + paths[y]; \\
15 \indent \indent end \\
16 \indent \indent if dist[y] = $\infty$ then \\
17 \indent \indent \indent INJECT(Q, y); \\
18 \indent \indent \indent dist[y] $\leftarrow$ dist[x] + 1; \\
19 \indent \indent end \\
20 \indent end \\
21 end \\
22 ; \\
23 return paths[v] \\


\end{document}

