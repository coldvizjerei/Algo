\documentclass{article}

\usepackage{amssymb}

\begin{document}
\begin{center}
\bf{\LARGE CS344: Design and Analysis of Computer Algorithms} \\*

\vspace{0.2in}
{\bf {\Large Homework 2}}
\end{center}

\vspace{.2in}

\noindent {\bf {\large Group Members: Stephen Kuo, Derek Mui}}

\vspace{.2in}
\noindent 1.11) Is $4^{1536} - 9^{4824}$ divisible by 35? \\
\vspace{.1in}
{\bf Answer:} \\
\indent $4^{1536} = (4^3)^{512} = 64^{512} = (35 + 29)^{512}$ \\
\indent Since 35 is divisible by 35, we can continue with $29^{512}$ \\
\indent $29^{512} = (29^2)^{256} = 841^{256} = (840 + 1)^{256}$ \\
\indent Since 840 is divisible by 35, we are left with $1^{256}$ mod 35 \\

\indent $9^4824 = (9^2)^{2412} = 81^{2412} = (70 + 11)^{2412}$ \\
\indent Since 70 is divisible by 35, we can continue with $11^{2412}$ \\
\indent $(11^3)^{804} = 1331^{804} = (1330 + 1)^{804}$ \\
\indent Since 1330 is divisible by 35, we are left with $1^{804}$ mod 35 \\

\indent ($1^{256}$ mod $35) - (1^{804}$ mod $35) = 0$ \\
\indent $\therefore$ Yes it is divisible by 35

\vspace{.3in}
\noindent 1.12) What is $2^{2^{2006}}$ mod 3 \\
\vspace{.1in}
{\bf Answer:} \\
\indent $2^{2^{2006}} =  4^{2006} = 4^{2^{1003}} = 16^{1003 } = (15 + 1)^{1003}$ \\
\indent We know 15 is divisible by 3, so that leaves us with $1^{1003}$. \\
\indent $\therefore$ Answer is 1

\vspace{.3in}
\noindent 1.13) Is the difference of $5^{30,000}$ and $6^{123,456}$ a multiple of 31? \\
\vspace{.1in}
{\bf Answer:} \\
\indent $5^{30000} = 5^{3^P{10000}} = 125^{10000} = (124 + 1)^{10000}$ \\
\indent Since 124 is divisible by 31, we are left with $1^{10000} $ mod 31 \\

\indent $6^{123456} = 6^{2^61728} = 36^{61728} = (31 + 5)^{61728}$ \\
\indent Since 31 is divisible by 31, we are left with $5^{61728}$ \\
\indent $5^{61728} = 5^{3^{20576}} = 125^{20576} = (124 + 1)^{20576}$ \\
\indent Since 124 is divisible by 31, that leaves us with $1^{20576}$ \\

\indent ($1^{10000}$ mod 31) - ($1^{20576}$ mod 31) $= 0$ \\
\indent $\therefore$ Yes, the difference is a multiple of 31

\vspace{.3in}
\noindent 1.25) calculate $2^{125}$ mod 127 using any method you choose \\
\vspace{.1in}
{\bf Answer:} \\
\indent $2^{125} = (2^{119}* 2^6) = (2^{7^{17}}* 2^6) = (128^{17}* 2^6) = (127 + 1)^{17}*2^6$ \\
\indent Because 127 is divisible by 127, that leaves us with $1^{17}*2^6$ \\
\indent $1^{17}*2^6 = 1 * 2^6 = 64$ \\
\indent $\therefore$ Answer is 64

\vspace{.3in}
\noindent 1.33) Give an efficient algorithm to compute the least common multiple of two 
\indent n-bit numbers x and y, that is, the smallest number divisible by both x and 
\indent y. What is the running time of your algorithm as a function of n? \\
\vspace{.1in}
{\bf Answer:} \\
\indent $(x * y)$ (include the shiftings) If $x$ and $y$ are both $n$ bits, then there are $n$ \\
\indent intermediate rows. \\
\indent So, $O(n) + O(n) + . . . +O(n)$ done $(n - 1)$ times yields $O(n^2)$ \\
\indent Divide the same (include the shiftings) = $O(n^2)$ \\

\indent gcd$(x, y) = O(n^3)$ initially $n$-bit integers, base case. \\
\indent Each quadratic - time division reaches within 2$n$ recursive calls. \\
\indent $\therefore$ Total $= O(n^3)$

\vspace{.3in}
\noindent 1.39) Give a polynomial-time algorithm for computing $a^{b^c}$ mod $p$, given $a, b, c,$ 
\indent and prime $p$. \\
\vspace{.1in}
{\bf Answer:} \\
\indent temp

\vspace{.3in}
\noindent Problem) \\
\vspace{.1in}
{\bf Answer:} \\
\indent temp

\end{document}